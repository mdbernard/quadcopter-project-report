\section*{Physical System}

Due to symmetry about the $x$ and $y$ axes of the quadcopter, the vehicle's inertia matrix is defined as
\begin{equation}
    I = 
    \begin{bmatrix}
        I_{xx} & 0 & 0 \\
        0 & I_{yy} & 0 \\
        0 & 0 & I_{zz}
    \end{bmatrix} \quad ,
\end{equation} where $I_{xx} = I_{yy}$ due to symmetry. Each of the rotors induces a lifting force on the vehicle:
\begin{equation}
    f_{i} = k \omega_i^2
\end{equation}
and a torque about the rotor axis: $\tau_{M_{i}} = b \omega_i^2 + I_M \dot{\omega}_i ~,$ where $k$ is the rotor lift constant, $b$ is the rotor drag constant, and $I_M$ is the inertia moment of the rotor. The value of $\dot{\omega}_i$ is considered small, and so the torque equation is simplified to 
\begin{equation}
    \tau_{M_{i}} = b \omega_i^2 \quad .
\end{equation}

The general 6-DOF inertial-frame Force Equations are as follows:
\begin{align*}
    \dot{U} &= RV - QW - g_D\sin(\theta) + \dfrac{(X_A + X_T)}{m} \\
    \dot{V} &= -RU + PW + g_D\sin(\phi)\cos(\theta) + \dfrac{(Y_A + Y_T)}{m} \\
    \dot{W} &= QU - PV + g_D\cos(\phi)\cos(\theta) + \dfrac{(Z_A + Z_T)}{m} \quad ,
\end{align*}
where the variables are defined in Table 1 below. Since the flight speed of the quadcopter is low, the aerodynamic drag forces $X_A$, $Y_A$, and $Z_A$ can be considered negligible. If $X_A = Y_A = Z_A = 0$, the equations simplify to
\begin{align}
    \dot{U} &= RV - QW - g_D\sin(\theta) + \dfrac{X_T}{m} \\
    \dot{V} &= -RU + PW + g_D\sin(\phi)\cos(\theta) + \dfrac{Y_T}{m} \\
    \dot{W} &= QU - PV + g_D\cos(\phi)\cos(\theta) + \dfrac{Z_T}{m} \quad\quad .
\end{align}

Figures 2, 3, and 4 show the simplified equations % TODO: add equation numbers when finalized
above as implemented in Simulink. % TODO: add images here
Figure 5 shows how these blocks are connected at a high-level in a Force Equations block in Simulink.

The Kinematic Equations for the Euler Angles of the vehicle in the inertial frame are as follows:
\begin{align}
    \dot{\phi} &= P + \tan(\theta)(Q\sin(\phi) + R\cos(\phi)) \\
    \dot{\theta} &= Q\cos(\phi) - R\sin(\theta) \\
    \dot{\psi} &= \dfrac{Q\sin(\phi) + R\cos(\phi)}{\cos(\theta)} \quad ,
\end{align}
where $\psi$, $\phi$, $\theta$ are the Euler Angles about the $z$, $y$, and $x$ axes in the inertial frame, respectively. Figures 6, 7, and 8 show an implementation of these equations in Simulink. Figure 9 shows how these blocks are connected at a high-level in a Kinematic Equations block. % TODO: include figures here

The 6-DOF Moment Equations are as follows:
\begin{align*}
    \dot{P} &= \dfrac{I_{xz}[ J_{xx} - I_{yy} + I_{zz} ]PQ - [ I_{zz}(I_{zz} - I_{yy}) + I_{xy}^2 ]QR + I_{zz} l + I_{xz} n}{\Gamma} \\
    \dot{Q} &= \dfrac{(I_{zz} - I_{xx})PR - I_{xz}(P^2 - R^2) + m}{I_{yy}} \\
    \dot{R} &= \dfrac{[(I_{xx} - I_{yy})I_{xx} + I_{xz}^2]PQ - I_{xz}[I_{xx} - I_{yy} + I_{zz}]QR + I_{xz} l + I_{xx} n}{\Gamma} \quad ,
\end{align*}
where $P$, $Q$, and $R$ are the moments about the $x$, $y$, and $z$ axes in the inertial frame, respectively, and $\Gamma = I_{xx}I_{zz} - I_{xz}^2$. Examining the simplified inertia matrix in (1), the Moment Equations simplify to:
\begin{align}
    \dot{P} &= \dfrac{-I_{zz}(I_{zz} - I_{yy})QR + I_{zz} l}{\Gamma} \\
    \dot{Q} &= \dfrac{(I_{zz} - I_{xx})PR + m}{I_{yy}} \\
    \dot{R} &= \dfrac{[(I_{xx} - I_{yy})I_{xx}]PQ + I_{xx} n}{\Gamma} \quad .
\end{align}
\noindent Note here that $l$, $m$, and $n$ are the $x$, $y$, and $z$ components of $\vec{M}_{A,T}^i$ ($l$ and $m$ are not the length and mass values given in Table 1).

Figures 10, 11, and 12 show an implementation of equations % insert equation numbers here
in Simulink. Figure 13 shows how these blocks are connected at a high-level in a Moment Equations block.
% TODO: add figures here

We are concerned only with altitude control of the vehicle, not lateral control. As such, we use only the following Navigation Equation:
\begin{equation}
    \dot{h} = U\sin(\theta) - V\sin(\phi)\cos(\theta) - W\cos(\phi)\cos(\theta) \quad .
\end{equation}
Figure 14 shows an implementation of this equation in Simulink. Figure 15 shows how the Force Equations, Kinematic Equations, Moment Equations, and Navigation Equation blocks are connected at a high-level in a 6-DOF Model block in Simulink.