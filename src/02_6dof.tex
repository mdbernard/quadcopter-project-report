\section*{6 Degree of Freedom Model}

Due to symmetry about the $x$ and $y$ axes of the quadcopter, the vehicle's inertia matrix is defined as
\begin{equation}
    I = 
    \begin{bmatrix}
        I_{xx} & 0 & 0 \\
        0 & I_{yy} & 0 \\
        0 & 0 & I_{zz}
    \end{bmatrix} \quad ,
\end{equation} where $I_{xx} = I_{yy}$ due to symmetry.

The Cartesian components of force on the vehicle in the inertial frame are defined as follows:
\begin{align*}
    \dot{U} &= RV - QW - g_D\sin(\theta) + \dfrac{(X_A + X_T)}{m} \\
    \dot{V} &= -RU + PW + g_D\sin(\phi)\cos(\theta) + \dfrac{(Y_A + Y_T)}{m} \\
    \dot{W} &= QU - PV + g_D\cos(\phi)\cos(\theta) + \dfrac{(Z_A + Z_T)}{m} \quad .
\end{align*}
Rather than define all variables following each equation, all variables are defined in Table 1 at the end of this report. Since the flight speed of the quadcopter is low, the aerodynamic drag forces $X_A$, $Y_A$, and $Z_A$ can be considered negligible. If $X_A = Y_A = Z_A = 0$, the equations simplify to
\begin{align}
    \dot{U} &= RV - QW - g_D\sin(\theta) + \dfrac{X_T}{m} \\
    \dot{V} &= -RU + PW + g_D\sin(\phi)\cos(\theta) + \dfrac{Y_T}{m} \\
    \dot{W} &= QU - PV + g_D\cos(\phi)\cos(\theta) + \dfrac{Z_T}{m} \quad\quad .
\end{align}

Figures 2, 3, and 4 show the simplified equations (2), (3), and (4) above as implemented in Simulink. Figure 5 shows how these blocks are connected at a high-level in a Force Equations block in Simulink.

The Kinematic Equations for the Euler Angles of the vehicle in the inertial frame are as follows:
\begin{align}
    \dot{\phi} &= P + \tan(\theta)(Q\sin(\phi) + R\cos(\phi)) \\
    \dot{\theta} &= Q\cos(\phi) - R\sin(\theta) \\
    \dot{\psi} &= \dfrac{Q\sin(\phi) + R\cos(\phi)}{\cos(\theta)} \quad .
\end{align}
Figures 6, 7, and 8 show an implementation of these equations in Simulink. Figure 9 shows how these blocks are connected at a high-level in a Kinematic Equations block.

The general 6-DOF Moment Equations in the inertial frame are as follows:
\begin{align*}
    \dot{P} &= \dfrac{I_{xz}[ J_{xx} - I_{yy} + I_{zz} ]PQ - [ I_{zz}(I_{zz} - I_{yy}) + I_{xy}^2 ]QR + I_{zz} l + I_{xz} n}{\Gamma} \\
    \dot{Q} &= \dfrac{(I_{zz} - I_{xx})PR - I_{xz}(P^2 - R^2) + m}{I_{yy}} \\
    \dot{R} &= \dfrac{[(I_{xx} - I_{yy})I_{xx} + I_{xz}^2]PQ - I_{xz}[I_{xx} - I_{yy} + I_{zz}]QR + I_{xz} l + I_{xx} n}{\Gamma} \quad .
\end{align*}
Examining the simplified inertia matrix in equation (1), the Moment Equations simplify to:
\begin{align}
    \dot{P} &= \dfrac{-I_{zz}(I_{zz} - I_{yy})QR + I_{zz} l}{\Gamma} \\
    \dot{Q} &= \dfrac{(I_{zz} - I_{xx})PR + m}{I_{yy}} \\
    \dot{R} &= \dfrac{[(I_{xx} - I_{yy})I_{xx}]PQ + I_{xx} n}{\Gamma} \quad .
\end{align}
\noindent Note here that $l$, $m$, and $n$ are the $x$, $y$, and $z$ components of moment due to the propellers ($l$ and $m$ are not the length and mass values given in Table 1).

Figures 10, 11, and 12 show an implementation of the equations above in Simulink. Figure 13 shows how these blocks are connected at a high-level in a Moment Equations block.

We are concerned only with altitude control of the vehicle, not lateral control. As such, we use only the following Navigation Equation:
\begin{equation}
    \dot{h} = U\sin(\theta) - V\sin(\phi)\cos(\theta) - W\cos(\phi)\cos(\theta) \quad .
\end{equation}
Figure 14 shows an implementation of this equation in Simulink. The equation was multiplied by $-1$ in the implementation in order to reflect a positive height being upwards, which is opposite the convention in the equation. Figure 15 shows how the Force Equations, Kinematic Equations, Moment Equations, and Navigation Equation blocks are connected at a high-level in a 6-DOF Model block in Simulink.